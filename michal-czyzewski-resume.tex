\documentclass[10pt,a4paper]{article}

% include the `tex` instructions that takes care of loading packages and defining commands

% Copyright (c) 2012 Cies Breijs
%
% The MIT License
%
% Permission is hereby granted, free of charge, to any person obtaining a copy
% of this software and associated documentation files (the "Software"), to deal
% in the Software without restriction, including without limitation the rights
% to use, copy, modify, merge, publish, distribute, sublicense, and/or sell
% copies of the Software, and to permit persons to whom the Software is
% furnished to do so, subject to the following conditions:
%
% The above copyright notice and this permission notice shall be included in
% all copies or substantial portions of the Software.
%
% THE SOFTWARE IS PROVIDED "AS IS", WITHOUT WARRANTY OF ANY KIND, EXPRESS OR
% IMPLIED, INCLUDING BUT NOT LIMITED TO THE WARRANTIES OF MERCHANTABILITY,
% FITNESS FOR A PARTICULAR PURPOSE AND NONINFRINGEMENT. IN NO EVENT SHALL THE
% AUTHORS OR COPYRIGHT HOLDERS BE LIABLE FOR ANY CLAIM, DAMAGES OR OTHER
% LIABILITY, WHETHER IN AN ACTION OF CONTRACT, TORT OR OTHERWISE, ARISING FROM,
% OUT OF OR IN CONNECTION WITH THE SOFTWARE OR THE USE OR OTHER DEALINGS IN THE
% SOFTWARE.


% Some commands for making a LaTeX resume
% =======================================

% Commented ;)

% See the README.md file for more info



% \documentclass[10pt,a4paper]{article}  % i do this in the document itself


%%% LOAD AND SETUP PACKAGES

\usepackage[a4paper,margin=0.75in]{geometry}
\usepackage[utf8]{inputenc}
\usepackage{mdwlist}   % to finetue lists with a inline heading and indented content (see Experiences)
\usepackage{multicol}  % for multiple column text
\usepackage{relsize}   % for \textscale, which I prefer over \sc (small caps), see my \acr command
\usepackage[english]{babel}
\hyphenation{Some-long-word}

%\usepackage[pdftex]{hyperref}  % yups, URLs everwhere...
\usepackage{hyperref}  % yups, URLs everwhere...
\usepackage{xcolor}  % ... and color them links
\definecolor{dark-blue}{rgb}{0.15,0.15,0.4}
\hypersetup{colorlinks,linkcolor={dark-blue},citecolor={dark-blue},urlcolor={dark-blue}}

\usepackage{ifxetex}
\ifxetex
  \usepackage{fontspec}
  \setmainfont
    [ ExternalLocation ,
      Mapping          = tex-text ,
      Numbers          = OldStyle ,
      Ligatures        = {Common,Contextual} ,
      BoldFont         = texgyrepagella-bold.otf ,
      ItalicFont       = texgyrepagella-italic.otf ,
      BoldItalicFont   = texgyrepagella-bolditalic.otf ]
    {texgyrepagella-regular.otf}
  % Comment out the previous statement and uncomment the following line to use the
  % Linux Libertine font (it has nice lignatures).
  % Make sure to have the `ttf-linux-libertine` package installed on Ubuntu.
%  \setmainfont[Mapping=tex-text, Numbers=OldStyle, Ligatures={Common,Contextual}]{Linux Libertine O}
  \usepackage[protrusion]{microtype}  % needs an experimental and impposible to find package for xetex
\else
  \usepackage{tgpagella}  % this case we lack lower case numbers, ligatures and some typographic niceties
  \usepackage[expansion,protrusion]{microtype}
\fi



%%% DOCUMENT WIDE STYLING

\pagestyle{empty}
\setlength{\tabcolsep}{0em}
\xspaceskip7pt  % some more spacing between sentences (use "i.e.\ " or "with SQL\@. " in case of errors)


%%% CUSTOM COMMANDS

% main title (name) with subtitle (date)
\newcommand*\maintitle[2]{\noindent{\LARGE \textbf{#1}}\ \ \ \emph{#2}}

% title for the root sections (experience, education, etc) of the resume
\newcommand*\roottitle[1]{\subsection*{#1}\vspace{-0.3em}\nopagebreak[4]}

% acr command, to quickly mark acronyms for special formatting
\newcommand*\acr[1]{\textscale{.85}{#1}}

% pretty bullet (created from a much smaller centerdot), \sbull contains its spacing
\newcommand*\bull{\raisebox{-0.365em}[-1em][-1em]{\textscale{4}{$\cdot$}}}
\newcommand*\sbull{\ \ \bull \ \ }

% it seems not to work when simply using \parindent...
\newlength{\newparindent}
\addtolength{\newparindent}{\parindent}

% a double \parindent...
\newlength{\doubleparindent}
\addtolength{\doubleparindent}{\parindent}
\addtolength{\doubleparindent}{\parindent}

% indentsection style, used for sections that aren't already in lists
% that need indentation to the level of all text in the document
\newenvironment{indentsection}%
{\begin{list}{}%
  {\setlength{\leftmargin}{\newparindent}\setlength{\parsep}{0pt}\setlength{\parskip}{0pt}\setlength{\itemsep}{0pt}\setlength{\topsep}{0pt}}%
}
{\end{list}}

% headerrow command, used for a new employer
\newcommand{\headedsection}[3]{\nopagebreak[4]\begin{indentsection}\item[]\textscale{1.1}{#1}\hfill#2#3\end{indentsection}\nopagebreak[4]}

% subheaderrow command, used for a new position
\newcommand{\headedsubsection}[3]{\nopagebreak[4]\begin{indentsection}\item[]\textbf{#1}\hfill\emph{#2}#3\end{indentsection}\nopagebreak[4]}

% body text (indented)
\newcommand{\bodytext}[1]{\nopagebreak[4]\begin{indentsection}\item[]#1\end{indentsection}\pagebreak[2]}

% \vspace variaties
\newcommand{\breakvspace}[1]{\pagebreak[2]\vspace{#1}\pagebreak[2]}
\newcommand{\nobreakvspace}[1]{\nopagebreak[4]\vspace{#1}\nopagebreak[4]}

% \spacedhrule a horizontal line with some vertical space before and after it
\newcommand{\spacedhrule}[2]{\breakvspace{#1}\hrule\nobreakvspace{#2}}

% \inlineheadsection command, used for a new employer
\newcommand{\inlineheadsection}[2]{\begin{basedescript}{\setlength{\leftmargin}{\doubleparindent}}\item[\hspace{\newparindent}\textbf{#1}]#2\end{basedescript}\vspace{-1.7em}}

% apo command, for an apostrophe that looks good on old style nums
\newcommand{\apo}{\raisebox{-.18ex}{'}{\hspace{-.1em}}}

% non space that allows line breaks
\newcommand*{\nsp}{\hskip0pt}

%%% MORE SPECIFIC COMMANDS

% CPP command (found it in some corner of the internet and decided to use it)
\newcommand{\CPP}{C\nolinebreak[4]\hspace{-.04em}\raisebox{.20ex}{\footnotesize\bf++}}



% % these are in the document itself:
%
% \begin{document}
% ...the document text...
% \end{document}

% NEW NEW COMMANDS
% footer
\newcommand{\footer}[1]{\noindent{\footnotesize#1}}

\begin{document}  % begin the content of the document
\sloppy  % this to relax whitespacing in favour of straight margins

\maintitle{Michał Czyżewski}{October 31, 1988}  % title on top of the document

\nobreakvspace{0.3em}  % add some page break averse vertical spacing

% \noindent prevents paragraph's first lines from indenting
% \mbox is used to obfuscate the email address
% \sbull is a spaced bullet
% \href well..
% \\ breaks the line into a new paragraph
\noindent\href{mailto:me@czesiek.net}{me\mbox{}@\mbox{}czesiek.net}\sbull
\textsmaller{+}48 604 464 147\sbull
\href{http://www.linkedin.com/in/mczyzewski}{www.linkedin.com/in/mczyzewski}
\\
ks.\ Jana Sitnika 6 m 13\sbull
01-410\sbull
Warsaw\sbull
Poland


\spacedhrule{1.4em}{0.1em}  % a horizontal line with some vertical spacing before and after

\roottitle{Experience}

\headedsection
  {\href{http://brama.elka.pw.edu.pl}{BRAMA Mobile Technology Laboratory}}
  {} {%
  \headedsubsection
    {hacker}
    {Oct \apo08 -- current}
    {\bodytext{
            \emph{AVStreaming} (since Oct \apo09) -- Coordination and on-site support for video streaming of various meetings/conferences (including local NetTuesday meetings, Creative Commons Global Summit 2011 and the visit of Richard Stallman in 2013).

      \emph{Linux w Bramie} (since Oct \apo08) -- Coordination, server and website administration for Free Software meetings/workshops.

      Technical support for the events co-organized with the Warsaw
      University of Technology.
%      (Sep \apo10: \emph{Networks 2010} international telecomunication symposium; \
%      Jun \apo09: \emph{Mobile Open Platform Seminars} event; among others)

      \emph{(references available upon request)}
    }}
}

\headedsection
  {\href{http://infullmobile.com}{inFullMobile}}
  {} {%
  \headedsubsection
    {web/backend developer}
    {Jul \apo10 -- Dec \apo11}
    {\bodytext{Design, implementation and deployment of RESTful APIs for mobile phones.

    Implementation and modification of websites (including mobile-oriented).

    \emph{Apple Push Notification Service} scripting.

    Deployment on \emph{Amazon Web Services} cloud solution.}}
  \headedsubsection
    {system administrator}
    {Jun \apo11 -- Dec \apo11}
    {\bodytext{Creation and maintenance of in-house development infrastructure based on GNU/Linux Gentoo \
    featuring Trac and Gitolite along with various project deployments.

    Administration of company networks.}}
}


\spacedhrule{1.0em}{0.1em}

\roottitle{Education}

\headedsection
  {Warsaw University of Technology, Faculty of Electronics and Information Technology}
  {} {%
  \headedsubsection
    {Electronics and Computer Engineering}
    {2007 -- 2010} {}
}


\spacedhrule{1.0em}{0.1em}

\roottitle{Skills}

\inlineheadsection  % special section that has an inline header with a 'hanging' paragraph
  {Technical specialties:}
  {Software design and implementation in PHP, Python, Bash, C/\CPP, \acr{HTML+CSS}, JavaScript.
  I am interested in functional programming and passionate about \acr{UI}/\acr{UX} design.

  \textbf{Libraries/frameworks:} Bottle, Smarty, jQuery, Mustache.

  \textbf{Version control systems:} Git, Bazaar, SVN.

  \textbf{Database design/administration:} Postgre\acr{SQL}, My\acr{SQL}, sqlite,
  pgAdmin, My\acr{SQL} Workbench, basic knowledge of
  No\acr{SQL}/key-value store systems.

  \textbf{Linux administration:} Apache, lighttpd, WSGI, exim, dovecot; solid
  knowledge of Gentoo and Debian GNU/Linux.

  \textbf{Virtualization technologies:} VirtualBox, EC2.
  }

%\parskip

\inlineheadsection
  {Natural languages:}
  {Polish \emph{(native)}, English \emph{(full professional proficiency)} and Japanese \emph{(beginner)}.}


\spacedhrule{2.0em}{0.1em}

\roottitle{Interests}

\bodytext{Cryptography, Free Software, Free Culture.

Japanese language, culture and cuisine. Cooking.

Cycling, running, sailing and hiking.}


\spacedhrule{1.0em}{0.5em}

\footer{I hereby agree for my personal data, included in my job application, to \
be processed in line with the needs of recruitment, in accordance with the Law \
on Personal Data Protection of 29 August 1997 (Law Gazette from 2002, No.\ 101, \
heading 926, as amended).}


\end{document}
